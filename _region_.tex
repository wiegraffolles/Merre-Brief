\message{ !name(Setting Brief.tex)}%%%%%%%%%%%%%%%%%%%%%%%%%%%%%%%%%%%%%%%%%
% Short Sectioned Assignment
% LaTeX Template
% Version 1.0 (5/5/12)
%
% This template has been downloaded from:
% http://www.LaTeXTemplates.com
%
% Original author:
% Frits Wenneker (http://www.howtotex.com)
%
% License:
% CC BY-NC-SA 3.0 (http://creativecommons.org/licenses/by-nc-sa/3.0/)
%
%%%%%%%%%%%%%%%%%%%%%%%%%%%%%%%%%%%%%%%%%

%----------------------------------------------------------------------------------------
%	PACKAGES AND OTHER DOCUMENT CONFIGURATIONS
%----------------------------------------------------------------------------------------

\documentclass[paper=a4, fontsize=11pt]{scrartcl} % A4 paper and 11pt font size

\usepackage[T1]{fontenc} % Use 8-bit encoding that has 256 glyphs
\usepackage{multicol} % For dual columns
\usepackage{parskip} % For setting spacing between paragraphs
\usepackage{titlesec} % For formatting section headings

\usepackage[english]{babel} % English language/hyphenation
\usepackage{fontspec} % For loading fonts
\defaultfontfeatures{Mapping=tex-text}
\setmainfont{Linux Libertine} % Main document font
\addfontfeature{Ligatures=Historical}
\newfontfamily\medievalsharp{MedievalSharp}

\usepackage{lipsum} % Used for inserting dummy 'Lorem ipsum' text into the template

\usepackage{sectsty} % Allows customizing section commands
\allsectionsfont{\centering \medievalsharp\uppercase} % Make all sections centered, with the Medieval Sharp font and all caps

\usepackage{fancyhdr} % Custom headers and footers
\pagestyle{fancyplain} % Makes all pages in the document conform to the custom headers and footers
\fancyhead{} % No page header - if you want one, create it in the same way as the footers below
\fancyfoot[L]{} % Empty left footer
\fancyfoot[C]{\thepage} % Page numbering for center footer
\fancyfoot[R]{} % Empty right footer
\renewcommand{\headrulewidth}{0pt} % Remove header underlines
\renewcommand{\footrulewidth}{0pt} % Remove footer underlines
\setlength{\headheight}{13.6pt} % Customize the height of the header
\usepackage{lettrine}


\setlength\parindent{0pt} % Removes all indentation from paragraphs - comment this line for an assignment with lots of text

\setlength\parskip{8pt}

%----------------------------------------------------------------------------------------
%	TITLE SECTION
%----------------------------------------------------------------------------------------

\newcommand{\horrule}[1]{\rule{\linewidth}{#1}} % Create horizontal rule command with 1 argument of height

\title{	
\normalfont \normalsize 
% \textsc{Doshisha University, Faculty of Global and Regional Studies, English Division} \\ [25pt] % Your university, school and/or department name(s)
\horrule{0.5pt} \\[0.4cm] % Thin top horizontal rule
\begingroup
\medievalsharp
\huge \uppercase{Merre Campaign Setting Brief} \\ % The assignment title
\endgroup
\horrule{2pt} \\[0.5cm] % Thick bottom horizontal rule
}

\author{} % Your name

\date{} % Today's date or a custom date

\begin{document}

\message{ !name(Setting Brief.tex) !offset(151) }
 half-orc adventurers have achieved great deeds in the civilized lands, and there are more than a few noble houses whose distant ancestors are said to be conquering half-orc heroes (although this is not mentioned in polite conversation). 

\end{multicols}

\section*{Religion}

\horrule{0.5pt} \\[0.4cm] % Thin top horizontal rule

\begin{multicols}{2} % Begin double column layout

\subsection*{The Gods}

\lettrine[lines=2]{\medievalsharp M}{ost} inhabitants of Merre are polytheists, worshipping at traditional shrines to local gods, or worshipping according to a god's reputation when travelling. The religion of Merre is a confusion of gods and goddesses who vary from shrine to shrine, and locality to locality. From one town to the next two gods may have very similar characteristics but are claimed to be different personages. Priests and scholars fiercely debate whether these gods are different aspects of a smaller number of gods or if they are aspects of one god, or if they are different altogether, but this does not matter much to worshipers, who are interested primarily in what the gods are reputed to \textit{do} and what festivals they preside over. Gods of the harvest or fertility, or love, or natural disasters are worshipped because of their power over worldly affairs, and it is mainly the priesthood who concern themselves with the particulars of their divine biography.

Most cults in Merre originate from the ancestral cults of the aristocracy, and the high priesthood are typically drawn from the ranks of these aristocratic families. Shrine musicians, dancers, and servants are typically of more common stock. Outside of these established cults are those of folk religion, whose priesthood, if any exists at all, are typically comprised of ascetics living in the wild, who occasionally descend into civilization to perform holy rites. Some of these holy men may be druids instead of priests.

\subsection*{The Imperial Cults}

\lettrine[lines=2]{\medievalsharp O}{ut} of the chaotic mass of gods and goddesses worshipped across Merre, only those cults associated with the old Imperial families of the Grans Imperium are so widespread as to be of particular note. They are covered briefly below.

\subsubsection*{Alera}

Alera is a goddess of love. She is best known for the story of how she was courted by Errif, god of justice, and Bled, god of war. She let them court her for 10,000 years, until out of desperation they appealed to her father Or to make her decide which to marry. Or commanded her to decide by nightfall which to marry, and out of spite she cut herself in half rather than decide. Errif and Bled then came before Or carrying her corpse, and wept a sea of tears while they beseeched him to restore her to life. 

Or went to his brother Sarron, king of hell, and asked him to return her spirit from his land. Unfortunately Sarron had been so taken with the beauty of her soul that he had installed it in a masterfully crafted golden statue and taken her for his own bride. This form of Alera is known as Alera-Gravda, and is worshipped by painters and sculptors as the ideal work of artifice. Sarron refused his brother's request and told him that she would not return from the land of the dead. Or flew into a rage at Sarron's defiance and attacked Sarron, beginning a battle that lasted another 10,000 years and shook the earth and heavens. Finally Or bested Sarron and his armies, and tore Sarron's heart from his chest. With the power of Sarron's heart in hand, Or smote the golden statue of Alera and restored her soul to the world of the living. He then used the last of Sarron's power to split her soul in twain and out of the two halves of her corpse created wives for both Errif and Bled. 

Errif's wife is known as Alera-Eff, goddess of chastity, and Bled's wife is known as Alera-Olea, goddess of lust. Errif conceived no children by his wife, but Bled conceived one son by Alera-Olea, who was none other than Sarron reborn. The reborn child Sarron is known as Isher, and is the god of mischief, so known because before maturing and returning to his home in hell he spent five millennia tormenting his father and grandfather with tricks to punish them for his earlier murder. Temples to Alera typically depict her in both her aspects as Alera-Eff and Alera-Olea as statues staring down at the worshipper. Some temples also depict her in her unified aspect or as Alera-Gravda, but this is less common.


\message{ !name(Setting Brief.tex) !offset(154) }

\end{document}